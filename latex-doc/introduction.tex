\chapter{Introduction}
\label{sec:intro}
The U.S. offshore wind gross resource potential is vast, and often in
close proximity to densely-populated coastal load centers. In many
U.S. coastal areas, particularly in California and Hawaii, water depths
favor the deployment of floating over fixed-bottom offshore wind
technology. However, floating offshore wind technology is at a nascent
stage of development, and this technological immaturity means that
current designs are more expensive over their lifetime than their
fixed-bottom or land-based counterparts.

The first phase of floating offshore turbine designs married offshore
drilling platform substructures with derivative versions of on-shore
turbines and minor controller modifications to ensure operational
viability.  Since these designs were not engineered or optimized from a
systems or lifetime cost perspective, continuing this approach for the
next generation is not likely to provide a path to achieve cost
competitiveness. Nevertheless, prior analysis suggests that floating
wind turbines could eventually achieve cost competitiveness due to
better mass production opportunities, near-shore assembly to avoid heavy
lift installation vessels, and better wind resources in deeper
waters.

This document serves as both a User Manual and Theory Guide for
\textit{FloatingSE}, a key ingredient in realizing a system engineering
framework for floating offshore wind energy. \textit{FloatingSE} is one
of multiple modules within the Wind-Plant Integrated System Design and
Engineering Model (WISDEM).  WISDEM is a set of integrated modules that
creates a virtual, vertically integrated wind plant. The models use
engineering principles for conceptual design and preliminary analysis,
and link to financial modules for LCOE estimation.  The WISDEM modules,
including \textit{FloatingSE}, are built around the OpenMDAO library, an
open-source high-performance computing platform for systems analysis and
multidisciplinary optimization, written in Python \citep{openmdao}.

Due to the structure of OpenMDAO, and modular design of WISDEM, individual
modules can be exercised individually or in unison for turbine or plant
level studies.  This feature also applies to \textit{FloatingSE}, in
that module-specific optimizations of a floating substructure and its
anchoring and mooring systems can be executed while treating the rest
of the turbine, plant, and operational strategy as static.
Alternatively, \textit{FloatingSE} can be linked to other WISDEM modules
to execute turbine-level and/or plant-level tradeoff and optimization
studies.

This document presents an overview of the contents in
\textit{FloatingSE} in Section \ref{sec:package} and of the
parameterization of the substructure geometry in Section
\ref{sec:geom}.  With this foundation, the underlying theory of
\textit{FloatingSE's} methodology is explained in Section
\ref{sec:theory}.  This helps to understand the analysis execution flow
described in Section \ref{sec:exec} and the setup of design variables
and constraints for optimization problems in Section \ref{sec:opt}.
Finally, some discussion of how \textit{FloatingSE} can be linked to
other WISDEM modules via a high-level OpenMDAO Group is described in
Section \ref{sec:other}.
