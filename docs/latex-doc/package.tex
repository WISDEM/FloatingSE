\chapter{Package Documentation}
\label{sec:package}
\textit{FloatingSE} is a module within the larger WISDEM project,
developed primarily by engineers at the National Renewable Energy
Laboratory (NREL).  \textit{FloatingSE} is open-source and the project
repository is housed on GitHub at
\url{https://github.com/WISDEM/FloatingSE}.

\section{Package Files}
The files that comprise the \textit{FloatingSE} package are found in the
\texttt{src/floatingse} directory in accordance with standard Python
package conventions.  In addition to these files, there are also
affiliated unit tests that probe the behavior of individual functions.
These are located in the \texttt{test} sub-directory.  A summary of all
package files is included in Table \ref{tbl:package}.

\begin{table}[htbp] \begin{center}
    \caption{File contents of the \texttt{src/floatingse} Python package
      within \textit{FloatingSE}.}
    \label{tbl:package}
{\footnotesize
  \begin{tabularx}{\textwidth}{ l l X } \hline
    \textbf{File name} & \textbf{Unit test file} & \textbf{Description} \\ \hline \hline
\mytt{floating.py} & \mytt{floating\_PyU.py} & Top level \textit{FloatingSE} OpenMDAO Group.\\
\mytt{column.py} & \mytt{column\_PyU.py} & OpenMDAO Components and Group calculating
  mass, buoyancy, and static stability of vertical frustum columns.\\
\mytt{loading.py} & \mytt{loading\_PyU.py} &OpenMDAO Components and Group for
  Frame3DD analysis of complete turbine structure and final summation of
  mass and displacement.\\
\mytt{map\_mooring.py} & \mytt{mapMooring\_PyU.py} &Mooring analysis
using pyMAP++ module.\\
\mytt{substructure.py} & \mytt{substructure\_PyU.py} &Final buoyancy and stability checks of
  the substructure.\\
&\mytt{package\_PyU.py} & Convenience aggregator of all unit test files.\\
\mytt{instance/floating\_instance.py} && Parent class controlling
  optimization drivers, constraints, and visualization. \\
\mytt{instance/spar\_instance.py} && Spar example implementing design
  parameters and visualization.\\
\mytt{instance/semi\_instance.py} && Semisubmersible example implementing
  design parameters and visualization.\\
\mytt{instance/tlp\_instance.py} && Tension leg platform example implementing
  design parameters and visualization.\\
\hline \end{tabularx}
}
\end{center} \end{table}

\section{Dependencies}
\subsection{Python Dependencies}
\textit{FloatingSE} directly loads python modules from NumPy, SciPy,
OpenMDAO, and other WISDEM modules.  Each of these packages have
dependencies in their own right.  For visualization, there are optional
calls to Matplotlib and Mayavi, which is built on VTK.  All of the
python dependencies are free and open-source.

\subsection{WISDEM Dependencies}
The basic functioning of \textit{FloatingSE} relies heavily on
\textit{CommonSE}, another WISDEM module that has general utilities
applicable across the WISDEM suite.  To do a structural analysis on the
entire floating turbine structure, some OpenMDAO Components from
\textit{TowerSE} are also called directly.  In turn, other utilities
managed within WISDEM are also called, such as \textit{Akima} (for
differentiable splines), \textit{pyFrame3DD} (for structural analysis),
and \textit{pyMAP} (for mooring analysis).  Note that for these other
WISDEM modules, additional Python packages are required, as well as C
and Fortran compilers.
