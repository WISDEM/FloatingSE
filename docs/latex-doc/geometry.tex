\chapter{Geometry}
\label{sec:geom}
This section describes the variables and methods used to parameterize
the substructure geometry in \textit{FloatingSE}.  Typically,
substructure designs have fallen into three classical regimes, which are
shown in Figure \ref{fig:archetype}, each of which attains static stability
through different physical mechanisms.  A spar derives its stability from a
deep drafted ballast.  A semisubmersible derives its stability from
distributed waterplane area, achieved with offset columns spread evenly
around a main column or central point.  A tension leg platform (TLP) uses taut
mooring lines for its stability.

\begin{figure}[htbp]
  \begin{center}
    \includegraphics[width=3.75in]{figs/archetypes.pdf}
    \caption{Three classical designs for floating turbine substructures.}
    \label{fig:archetype}
  \end{center}
\end{figure}

Similar to \citet{karimi2017}, care was taken to parameterize the
substructure in a general manner, so as to be able to use the same set
of design variables to describe spars, semisubmersibles, TLPs, and
hybrids of those archetypes.  The intent is that this modular approach
to substructure definition will enable rapid analysis of the majority of
designs currently proposed by the floating wind development community,
whether classical or novel in nature.  Furthermore, generalizing the
substructure definition also empowers the optimization algorithm to
search a broad tradespace more efficiently by moving fluidly from one
region to another.

With that intent in mind, the general configuration of a spar-type
substructure is shown in Figure \ref{fig:diagram}, with nomenclature
borrowed from the field of naval architecture.  A semisubmersible
configuration would have a similar diagram, but with multiple offset
columns connected with pontoon elements.  A TLP might look similar to a
spar or semisubmersible, with taut mooring lines instead of the catenary
ones shown.

\begin{figure}[htb]
  \begin{center}
    \includegraphics[width=5in]{figs/diagram.pdf}
    \caption{Geometry parameterization with common wind turbine and
      naval architecture conventions.}
    \label{fig:diagram}
  \end{center}
\end{figure}

\section{Tapered Cylinders (Vertical Frustums)}
A number of typical floating substructure designs, such as the spar or
semisubmersible, contain vertically oriented columns.  In
\textit{FloatingSE}, these columns are
assumed to have a circular cross-section making them, formally, vertical
frustums.  These frustums are assumed to be ring-stiffened to support
the buckling loads inherent in a submerged support structure.  The
number of columns, their geometry, and the ring stiffeners are
parameterized in the \textit{FloatingSE} module according to the
diagrams in Figures \ref{fig:diagram} and \ref{fig:column}.  The main
column is assumed to be centered at $(x=0, y=0)$, directly underneath the
turbine tower (note that off-centered turbines are not yet supported).
Other columns are referred to as \textit{offset} columns, and are
assumed to be evenly spread around the main column.  The material of the
vertical columns is currently assumed to be ASTM 992 steel.
Future developments will include the option to select one of multiple
material options for each section in each cylinder.

\begin{figure}[htb]
  \begin{subfigure}[b]{0.38\linewidth}
    \centering \includegraphics[width=2.2in]{figs/colGeom.pdf}
    \caption{Vertical column of frustums}
  \end{subfigure}
  \begin{subfigure}[b]{0.29\linewidth}
    \centering \includegraphics[width=1.8in]{figs/stiffenerCut.pdf}
    \caption{Vertical cross-section}
  \end{subfigure}
  \begin{subfigure}[b]{0.29\linewidth}
    \centering \includegraphics[width=1.8in]{figs/stiffenerZoom.pdf}
    \caption{Ring stiffener geometry}
  \end{subfigure}
  \caption{Vertical frustum geometry parameterization.}
  \label{fig:column}
\end{figure}

The variables that set the geometry of the main and offset columns are listed in Table \ref{tbl:main.ar}.  Two
additional variables are also included that describe the placement of
the offset columns within the substructure configuration.
%
\begin{table}[htbp] \begin{center}
    \caption{Variables specifying the main column geometry within \textit{FloatingSE}.}
    \label{tbl:main.ar}
{\footnotesize
  \begin{tabular}{ l l c l } \hline
    \textbf{Variable} & \textbf{Type} & \textbf{Units} & \textbf{Description} \\
    \mytt{main\_section\_height} & Float array ($n_s$) & $m$& Height of each section \\
    \mytt{main\_outer\_diameter} & Float array ($n_s+1$) & $m$&Diameter at each section node (linear lofting between) \\
    \mytt{main\_wall\_thickness} & Float array ($n_s+1$) & $m$&Wall thickness at each section node (linear lofting between) \\
    \mytt{main\_freeboard} & Float scalar & $m$&Design height above waterline \\
    \mytt{offset\_section\_height} & Float array ($n_s$) & $m$& Height of each section \\
    \mytt{offset\_outer\_diameter} & Float array ($n_s+1$) & $m$&Diameter at each section node (linear lofting between) \\
    \mytt{offset\_wall\_thickness} & Float array ($n_s+1$) & $m$&Wall thickness at each section node (linear lofting between) \\
    \mytt{offset\_freeboard} & Float scalar & $m$&Design height above waterline \\
    \mytt{number\_of\_offset\_columns} & Integer scalar && Number of offset columns in substructure (for spar set to 0)\\
    \mytt{radius\_to\_offset\_column} & Float scalar &$m$& Centerline of main.column to centerline of offset column\\
  \hline \end{tabular}
}
\end{center} \end{table}

\subsection{Discretization}
To allow for varying geometry parameters along the length of
substructure columns, the larger components are divided into sections.
The user may specify the number of overall sections, $n_s$ and the
geometry of each section.  Some of the geometry parameters are tied to
the nodes that bracket each section, such as column diameter and wall
thickness, with linear variation between each node.  Other parameters
are considered constant within each section, such as the spacing between
ring stiffeners.  The number of sections should resemble the physical
number of cans or sections used in the manufacturing of the real
article.

\subsection{Stiffeners}
The ring stiffener geometry is depicted in Figure \ref{fig:column}b--c
with geometry variables listed in Table \ref{tbl:stiffvar}
%
\begin{table}[htbp] \begin{center}
    \caption{Variables specifying the stiffener geometry within \textit{FloatingSE}.}
    \label{tbl:stiffvar}
{\footnotesize
  \begin{tabular}{ l l c l } \hline
    \textbf{Variable} & \textbf{Type} & \textbf{Units} & \textbf{Description} \\
    \mytt{main\_stiffener\_web\_height} & Float array ($n_s$) &$m$& Stiffener web height for each section \\
    \mytt{main\_stiffener\_web\_thickness} & Float array ($n_s$) &$m$& Stiffener web thickness for each section\\
    \mytt{main\_stiffener\_flange\_width} & Float array ($n_s$) &$m$& Stiffener flange width for each section\\
    \mytt{main\_stiffener\_flange\_thickness} & Float array ($n_s$) &$m$& Stiffener flange thickness for each section\\
    \mytt{main\_stiffener\_spacing} & Float array ($n_s$) &$m$& Stiffener spacing for each section\\
  \hline \end{tabular}
}
\end{center} \end{table}

\subsection{Material Properties}
The material of the vertical columns is currently assumed to uniformly
be ASTM 992 steel.  Future developments will include the option to
select one of multiple material options for each section in each
cylinder.  Currently, to globally change to a different material, use
the variables listed in Table \ref{tbl:materialvar}.
%
\begin{table}[htbp] \begin{center}
    \caption{Variables specifying the material properties within \textit{FloatingSE}.}
    \label{tbl:materialvar}
{\footnotesize
  \begin{tabular}{ l l c l } \hline
    \textbf{Variable} & \textbf{Type} & \textbf{Units} & \textbf{Description} \\
    \mytt{material\_density} & Float scalar & $kg/m^3$& Mass density (assumed steel) \\
    \mytt{E} & Float scalar & $N/m^2$& Young's modulus (of elasticity) \\
    \mytt{G} & Float scalar & $N/m^2$& Shear modulus \\
    \mytt{yield\_stress} & Float scalar & $N/m^2$& Elastic yield stress \\
    \mytt{nu} & Float scalar && Poisson's ratio ($\nu$)\\
  \hline \end{tabular}
}
\end{center} \end{table}

\subsection{Ballast}
Stability of substructure columns with long drafts can be enhanced by
placing heavy ballast, such as magnetite iron ore, at their bottom
sections.  The user can specify the density of the permanent ballast
added and the height of the ballast extent within the column. The variables that govern the implementation of the
permanent ballast and bulkhead nodes are listed in Table
\ref{tbl:ballastvar}.  Variable
ballast, as opposed to permanent ballast, is water that is added or
removed above the permanent ballast to achieve neutral buoyancy as the
operating conditions of the turbine change.  A discussion of variable
water balance in the model is found in Section \ref{sec:static}.
%
\begin{table}[htbp] \begin{center}
    \caption{Variables specifying the ballast and bulkheads within \textit{FloatingSE}.}
    \label{tbl:ballastvar}
{\footnotesize
  \begin{tabular}{ l l c l } \hline
    \textbf{Variable} & \textbf{Type} & \textbf{Units} & \textbf{Description} \\
    \mytt{permanent\_ballast\_density} & Float scalar & $kg/m^3$& Mass density of ballast material \\
    \mytt{main\_permanent\_ballast\_height} & Float scalar & $m$& Height above keel for permanent ballast \\
    \mytt{main\_bulkhead\_thickness} & Float vector ($n_s+1$) &$m$& Internal bulkhead thicknesses at section interfaces\\
    \mytt{offset\_permanent\_ballast\_height} & Float scalar & $m$& Height above keel for permanent ballast \\
    \mytt{offset\_bulkhead\_thickness} & Float vector ($n_s+1$) &$m$& Internal bulkhead thicknesses at section interfaces\\
  \hline \end{tabular}
}
\end{center} \end{table}


\subsection{Buoyancy Tanks (and Heave Plates)}
Buoyancy tanks are modeled as a collar around the column and are not
subject the same taper or connectivity constraints as the frustum
sections.  They therefore offer added buoyancy without incurring as much
structural mass or cost.  Moreover, they can also serve to augment the
heave added mass like a plate.  The variables that govern
their configuration are listed in Table \ref{tbl:buoyheave}.  In addition to their diameter and
height, the user can adjust the location of the buoyancy tank from the
column base to the top. Buoyancy tanks can be added to either the main
and/or offset columns.
%
\begin{table}[htbp] \begin{center}
    \caption{Variables specifying the buoyancy tank geometry within \textit{FloatingSE}.}
    \label{tbl:buoyheave}
{\footnotesize
  \begin{tabular}{ l l c l } \hline
    \textbf{Variable} & \textbf{Type} & \textbf{Units} & \textbf{Description} \\
    \mytt{main\_buoyancy\_tank\_diameter} & Float scalar & $m$& Diameter of buoyancy tank / heave plate on main.column\\
    \mytt{main\_buoyancy\_tank\_height} & Float scalar & $m$& Height of buoyancy tank / heave plate on main.column\\
    \mytt{main\_buoyancy\_tank\_location} & Float scalar & & Location of buoyancy tank along main.column (0 for bottom, 1 for top)\\
    \mytt{offset\_buoyancy\_tank\_diameter} & Float scalar & $m$& Diameter of buoyancy tank / heave plate on offset column\\
    \mytt{offset\_buoyancy\_tank\_height} & Float scalar & $m$& Height of buoyancy tank / heave plate on offset column\\
    \mytt{offset\_buoyancy\_tank\_location} & Float scalar & & Location of buoyancy tank along offliary column (0 for bottom, 1 for top)\\
  \hline \end{tabular}
}
\end{center} \end{table}


\section{Pontoons and Support Structure}
Many substructure designs include the use of pontoons that form a truss
to connect the different components, usually columns, together.  In this
model, all of the pontoons are assumed to have the identical thin-walled
tube cross section and made of the same material as the rest of the
substructure.  The truss configuration and the parameterization of the
pontoon elements is based on the members shown in Figure
\ref{fig:pontoon} with lettered labels.  The members are broken out into
the upper and lower rings connecting the offset columns ($B$ and $D$,
respectively), the upper and lower main-to-offset connections ($A$ and
$C$, respectively), the lower-base to upper-offset cross members ($E$),
and the V-shaped cross members between offset columns ($F$). The
variables that drive this parameterization are listed in Table
\ref{tbl:trussvar}.
%
\begin{figure}[htb]
  \begin{center}
    \includegraphics[width=4.5in]{figs/semi.pdf}
    \caption{Parameterization of truss elements in substructure.}
    \label{fig:pontoon}
  \end{center}
\end{figure}
%
\begin{table}[htbp] \begin{center}
    \caption{Variables specifying the pontoon and truss geometry within \textit{FloatingSE}.}
    \label{tbl:trussvar}
{\footnotesize
  \begin{tabular}{ l l c c l } \hline
    \textbf{Variable} & \textbf{Type} & \textbf{Figure \ref{fig:pontoon}} & \textbf{Units} & \textbf{Description} \\
    \mytt{pontoon\_outer\_diameter} & Float scalar & & $m$& Diameter of all pontoon/truss elements \\
    \mytt{pontoon\_wall\_thickness} & Float scalar & & $m$& Thickness of all pontoon/truss elements \\
    \mytt{main\_pontoon\_attach\_lower} & Float scalar & & $m$& Lower z-coordinate on main.where truss attaches \\
    \mytt{main\_pontoon\_attach\_upper} & Float scalar & & $m$& Upper z-coordinate on main.where truss attaches \\
    \mytt{upper\_attachment\_pontoons} & Boolean & A && Upper main.to-offset connecting pontoons\\
    \mytt{lower\_attachment\_pontoons} & Boolean & C && Lower main.to-offset connecting pontoons\\
    \mytt{cross\_attachment\_pontoons} & Boolean & E && Lower-Upper main.to-offset connecting cross braces\\
    \mytt{upper\_ring\_pontoons} & Boolean & B && Upper ring of pontoons connecting offset columns\\
    \mytt{lower\_ring\_pontoons} & Boolean & D && Lower ring of pontoons connecting offset columns\\
    \mytt{outer\_cross\_pontoons} & Boolean & F && Auxiliary ring connecting V-cross braces\\
  \hline \end{tabular}
}
\end{center} \end{table}


\section{Mooring Lines}
The mooring system is described by the number of lines, their geometry,
and their interface to the substructure.  The mooring diameter is set by
the user and determines the breaking load and stiffness of the chain,
via correlation, described in Section \ref{sec:theory}.  The mooring
lines attach to the substructure at the \textit{fairlead} distance below
the water plane, as shown in Figure \ref{fig:diagram}.  The lines can
attach directly to a substructure column or at a some offset from the
outer shell.  Note that bridle connections are not yet implemented in
the model.  The mooring lines attach to the sea floor at a variable
distance, the anchor radius, from the substructure centerline, also set
by the user.

By default, the mooring system is assumed to use a steel chain with drag
embedment anchors. Other mooring available for selection are nylon,
polyester, steel wire rope (IWRC) and fiber-core wire rope.  The only
alternative anchor type is currently suction pile anchors, but there are
plans to include gravity anchors as well.  The standard configuration
for TLPs is the use of taut nylon mooring lines with suction-pile
anchors.  The variables that control the mooring system properties are
listed in Table \ref{tbl:moorvar}.
%
\begin{table}[htbp] \begin{center}
    \caption{Variables specifying the mooring system within \textit{FloatingSE}.}
    \label{tbl:moorvar}
{\footnotesize
  \begin{tabular}{ l l c l } \hline
    \textbf{Variable} & \textbf{Type} & \textbf{Units} & \textbf{Description} \\
    \mytt{number\_of\_mooring\_connections} & Integer scalar && Number
    of evenly spaced mooring connection points\\
    \mytt{mooring\_lines\_per\_connection} & Integer scalar && Number of mooring lines at each connection point\\
    \mytt{mooring\_diameter} & Float scalar & $m$& Diameter of mooring line/chain \\
    \mytt{mooring\_line\_length} & Float scalar &$m$& Total unstretched line length of mooring line\\
    \mytt{fairlead\_location} & Float scalar && Fractional length from column bottom to mooring line attachment \\
    \mytt{fairlead\_offset\_from\_shell} & Float scalar & $m$ & Offset from shell surface for mooring attachment \\
    \mytt{anchor\_radius} & Float scalar & $m$& Distance from centerline to sea floor landing \\
    \mytt{mooring\_type} & Enumerated & & Options are CHAIN, NYLON, POLYESTER, FIBER, or IWRC\\
    \mytt{anchor\_type} & Enumerated & & Options are SUCTIONPILE or DRAGEMBEDMENT\\
  \hline \end{tabular}
}
\end{center} \end{table}



\section{Mass and Cost Scaling}
The mass of all components in the modeled substructure is captured
through calculation of each components' volume and multiplying by its material
density.  This applies to the frustum shells, the ring stiffeners, the
permanent ballast, the pontoons, and the mooring lines.
However, the model also acknowledges that the modeled substructure is
merely an approximation of an actual substructure and various secondary
elements are not captured.  These include ladders, walkways, handles,
finishing, paint, wiring, etc.  To account for these features en masse,
multipliers of component masses are offered as parameters for the user
as well.  Capital cost for all substructure components except the
mooring system is assumed to be a linear scaling of the components
masses.  For the mooring system, cost is dependent on the tension
carrying capacity of the line, which itself is an empirical function of
the diameter.  Default values for all mass and cost scaling factors are
found in Table \ref{tbl:factors}.  Cost factors are especially difficult to
estimate given the proprietary nature of commercial cost data, so
cost rates and estimates should be considered notional.
%
\begin{table}[htbp] \begin{center}
    \caption{Variables specifying the mass and cost scaling within \textit{FloatingSE}.}
    \label{tbl:factors}
{\footnotesize
  \begin{tabular}{ l l c r l } \hline
    \textbf{Variable} & \textbf{Type} & \textbf{Units} & \textbf{Default} & \textbf{Description} \\
    \mytt{bulkhead\_mass\_factor}     & Float scalar     &&1.0& Scaling for unaccounted bulkhead mass\\
    \mytt{ring\_mass\_factor}         & Float scalar     &&1.0& Scaling for unaccounted stiffener mass\\
    \mytt{shell\_mass\_factor}        & Float scalar     &&1.0& Scaling for unaccounted shell mass\\
    \mytt{column\_mass\_factor}       & Float scalar    &&1.05& Scaling for unaccounted column mass\\
    \mytt{outfitting\_mass\_fraction} & Float scalar    &&0.06& Fraction of additional outfitting mass for each column\\
    \mytt{ballast\_cost\_rate}        & Float scalar   & $USD/kg$&100& Cost factor for ballast mass \\
    \mytt{tapered\_col\_cost\_rate}    & Float scalar  & $USD/kg$&4,720& Cost factor for column mass \\
    \mytt{outfitting\_cost\_rate}     & Float scalar  & $USD/kg$&6,980& Cost factor for outfitting mass \\
    \mytt{mooring\_cost\_rate}        & Float scalar     &
    $USD/kg$&depends& Mooring cost factor (depends on diam and material) \\
    \mytt{pontoon\_cost\_rate}        & Float scalar   & $USD/kg$&6.5& Cost factor for pontoons \\
  \hline \end{tabular}
}
\end{center} \end{table}
