\chapter{Execution}
\label{sec:exec}
Executing \textit{FloatingSE} requires additional inputs beyond those of
the geometry definition described above in Section \ref{sec:geom}.
Other user inputs for the metocean and loading environment, and the
operational constraints, are required to evaluate the total mass, cost,
and code compliance.  These variables are described below, followed by
the sequence of \textit{FloatingSE} calculations.

\section{User Inputs}
The remaining input variables beyond those listed in Section
\ref{sec:geom} describe the metocean environment, the turbine geometry
and loading, as well as loading constraints.  The loading constraints
are more relevant in the context of optimization, so they are described
later in Section \ref{sec:opt}.  The remaining variables are explained
below.

\subsection{Metocean Environment}
The metocean condition is specified by the water and wind environment.
Geographical dependence is chiefly captured by specifying the water
depth.  The wave loading is parameterized by setting the significant
wave height, periodicity, and mean current speed (if any).  Physical
properties of the water, specifically the density and viscosity, are
also captured.  The user may also set the added mass constant used in
Equation \ref{eqn:morison} (Morison's equation).  All of these variables
cumulatively are in Table \ref{tbl:metocean}

\begin{table}[htbp] \begin{center}
    \caption{Variables specifying the wave environment within \textit{FloatingSE}.}
    \label{tbl:metocean}
{\footnotesize
  \begin{tabular}{ l l c l } \hline
    \textbf{Variable} & \textbf{Type} & \textbf{Units} & \textbf{Description} \\
    \mytt{water\_depth} & Float scalar & $m$& Distance to sea floor \\
    \mytt{Hs}        & Float scalar & $m$& Significant wave height \\
    \mytt{T}           & Float scalar & $s$& Wave period \\
    \mytt{cm}          & Float scalar && Added mass coefficient\\
    \mytt{Uc}          & Float scalar & $m/s$& Mean current speed \\
    \mytt{z0}          & Float scalar & $m$& z-coordinate of water line \\
    \mytt{water\_density}      & Float scalar & $kg/m^3$& Density of water \\
    \mytt{main.waveLoads.mu}  & Float scalar & $kg/m/s$& Viscosity of water \\
  \hline \end{tabular}
}
\end{center} \end{table}

Not that the some of these variables, especially the one setting water
viscosity, are awkwardly named.  This is due
to the need for OpenMDAO to have only one formal independent variable in any
high-level group.  Since \textit{FloatingSE} is intended to be married
with other WISDEM modules in simulation of a full turbine, the design
load cases must be specified at this higher-level group.  Thus,
\textit{FloatingSE} cannot declare independent variables relevant to the
load cases on its own, and must therefore use the variable names as
written in the module code.

The wind profile is specified by the user with a reference height and
velocity.  From there, wind speeds at other heights are determined using
a shear exponent, for power-law scaling, althrough logarithmic scaling
is available as well.  The physical properties of the air at the turbine
location must also be set.  As with the water-relevant variables, the
comment of awkwardly labeled variables applies.  Cumulatively, all of
the wind-related variables are in Table \ref{tbl:windvar}.

\begin{table}[htbp] \begin{center}
    \caption{Variables specifying the wind environment within \textit{FloatingSE}.}
    \label{tbl:windvar}
{\footnotesize
  \begin{tabular}{ l l c l } \hline
    \textbf{Variable} & \textbf{Type} & \textbf{Units} & \textbf{Description} \\
    \mytt{Uref}        & Float scalar & $m/s$& Wind reference speed \\
    \mytt{zref}        & Float scalar & $m$& Wind reference height \\
    \mytt{shearExp}    & Float scalar && Shear exponent in wind power law\\
    \mytt{beta}        & Float scalar & $deg$& Wind beta angle \\
    \mytt{main.windLoads.rho} & Float scalar & $kg/m^3$& Density of air \\
    \mytt{main.windLoads.mu}  & Float scalar & $kg/m/s$& Viscosity of air \\
  \hline \end{tabular}
}
\end{center} \end{table}

As mentioned in Section \ref{sec:theory}, only a single load case is
currently supported.  Future development will allow for optimization
around multiple load cases, each with their own metocean environment.


\subsection{Turbine Description}
To successfully analyze the entire load path through the substructure,
the user, or other WISDEM modules, must input the geometry and loading of
the wind turbine above the substructure.  The next component after the
substructure in the load path is the tower.  As a long, slender column,
the tower geometry parameterization is similar to that of the
substructure columns and has similar variable names, listed in Table
\ref{tbl:towervar}.

\begin{table}[htbp] \begin{center}
    \caption{Variables specifying the tower geometry within \textit{FloatingSE}.}
    \label{tbl:towervar}
{\footnotesize
  \begin{tabular}{ l l c l } \hline
    \textbf{Variable} & \textbf{Type} & \textbf{Units} & \textbf{Description} \\
    \mytt{hub\_height}              & Float scalar & $m$& Length from tower main.to top (not including freeboard) \\
    \mytt{tower\_section\_height}    & Float array ($n_s$) & $m$& Length of each tower section \\
    \mytt{tower\_outer\_diameter}    & Float array ($n_s+1$) & $m$& Diameter at each tower section node (linear lofting between) \\
    \mytt{tower\_wall\_thickness}    & Float array ($n_s+1$) & $m$& Diameter at each tower section node (linear lofting between) \\
    \mytt{tower\_buckling\_length}   & Float scalar & $m$& Tower buckling reinforcement spacing \\
    \mytt{tower\_outfitting\_factor} & Float scalar && Scaling for unaccounted tower mass in outfitting\\
  \hline \end{tabular}
}
\end{center} \end{table}


At the top of the load path, above the tower, is the rotor nacelle
assembly (RNA).  The RNA includes the blades, hub, shaft(s), gearbox,
generator, nacelle housing, etc.  All of these components are addressed
separately across multiple WISDEM modules, but \textit{FloatingSE} only
requires aggregate summations of the mass properties, forces, and
moments.  These cumulative variables are in Table \ref{tbl:rnavar}.

\begin{table}[htbp] \begin{center}
    \caption{Variables specifying the RNA geometry and loads within \textit{FloatingSE}.}
    \label{tbl:rnavar}
{\footnotesize
  \begin{tabular}{ l l c l } \hline
    \textbf{Variable} & \textbf{Type} & \textbf{Units} & \textbf{Description} \\
    \mytt{rna\_mass}   & Float scalar & $kg$& Mass \\
    \mytt{rna\_I}      & Float array (6) & $kg/m^2$& Moment of intertia (xx,yy,zz,xy,xz,yz) \\
    \mytt{rna\_cg}     & Float array (3) & $m$& Offset of RNA center of mass from tower top (x,y,z) \\
    \mytt{rna\_force}  & Float array (3) & $N$& Net force acting on RNA (x,y,z) \\
    \mytt{rna\_moment} & Float array (3) & $N*m$& Net moment acting on RNA (x,y,z) \\
    \mytt{yaw}         & Float scalar && Turbine yaw angle\\
  \hline \end{tabular}
}
\end{center} \end{table}


\section{Simulation Flow}
Once the input variables are completely specified, \textit{FloatingSE}
executes the analysis of the substructure.  Conceptually, the simulation
is organized by the flowchart in Figure \ref{fig:floatingse}.  From a
more technical perspective, \textit{FloatingSE} is an OpenMDAO Group, so
the analysis sequence is broken down by the sub-groups and
sub-components in the order that they are listed in Table
\ref{tbl:exec}.  In an OpenMDAO group, sub-groups and components are
given prefixes to aid in referring to specific variables.  The prefixes
used in \textit{FloatingSE} are also listed in Table \ref{tbl:exec}.
These prefixes also appear in some of the variable names listed above (e.g.,
\texttt{main.waveLoads.mu}) and will appear in the discussion of
optimization constraints in Section \ref{sec:opt}.

\begin{table}[htbp] \begin{center}
    \caption{Components and sub-groups within \textit{FloatingSE}.}
    \label{tbl:exec}
{\small
  \begin{tabular}{ l l l l } \hline
    &  \textbf{Prefix} & \textbf{Name} & \textbf{Description} \\ \hline\hline
    1) & \texttt{tow} & \textit{TowerLeanSE} & Discretization of tower
    geometry (but no analysis) \\
    2) & \texttt{main} & \textit{Column} & Discretization and API
    Bulletin 2U compliance of main.vertical column \\
    3) & \texttt{off} & \textit{Column} & Discretization and API
    Bulletin 2U compliance of offset columns \\
    4) & \texttt{sg} & \textit{SubstructureGeometry} & Geometrical constraints
    on substructure \\
    5) & \texttt{mm} & \textit{MapMooring} & Mooring system analysis via \textit{pyMAP} \\
    6) & \texttt{load} & \textit{FloatingLoading} & Structural analysis
    of complete floating turbine load path via \textit{pyFrame3DD} \\
    7) & \texttt{subs} & \textit{Substructure} & Static stability and final mass and cost summation for generic substructure \\
  \hline \end{tabular}
}
\end{center} \end{table}

\begin{figure}[htb]
  \begin{center}
    \includegraphics[width=5in]{figs/floatingse.pdf}
    \caption{Conceptual diagram of \textit{FloatingSE} execution.}
    \label{fig:floatingse}
  \end{center}
\end{figure}

Outputs are accumulated in each sub-group or component,
and they either become inputs to other components, become constraints
for optimization problems, become design variables for optimization
problems, or can simply be ignored.  Currently, a single execution of
FloatingSE takes only a handful of seconds on a modern laptop computer.


\section{Examples}
As mentioned in Section \ref{sec:package}, two files are meant as
analysis starting points, \texttt{sparInstance.py} and
\texttt{semiInstance.py}.  These files are encoded with default starting
configurations (from \citet{OC3} and \citet{OC4}, respectively).  They
also have ready configurations for optimization with design variables,
constraints, and solvers options.  However, due to the flexible and
object-oriented approach to programming these capabilities, some
complexity was introduced into the code and it is difficult to use as
simple examples.  For demonstration purposes, the spar and
semisubmersible examples from OC3 and OC4 are also provided at the
bottom of the \texttt{floating.py} file, and are copied below.  A
visualization of the geometries described by these examples is shown in
Figure \ref{fig:initial}

\begin{figure}[htb]
  \begin{subfigure}[b]{0.49\linewidth}
    \centering \includegraphics[width=\linewidth]{figs/spar-initial.pdf}
    \caption{OC3 Spar}
  \end{subfigure}
  \begin{subfigure}[b]{0.49\linewidth}
    \centering \includegraphics[width=\linewidth]{figs/semi-initial.pdf}
    \caption{OC4 Semisubmersible}
  \end{subfigure}\\
  \caption{Spar and semisubmersible examples in \textit{FloatingSE} taken from
    OC3\citep{OC3} and OC4\citep{OC4} projects.}
  \label{fig:initial}
\end{figure}

