\chapter{Introduction}
\label{sec:intro}
The complexity of the physics, economics, logistics, and operation of a
floating offshore wind turbine makes it well suited to systems-focused
solutions.  \textit{FloatingSE} is the floating substructure cost and
sizing module for NREL's Wind-Plant Integrated System Design and
Engineering Model (WISDEM) tool.  WISDEM is a set of integrated modules
that creates a virtual, vertically integrated wind plant. The models use
engineering principles for conceptual design and preliminary analysis,
and link to financial modules for LCOE estimation.

The WISDEM modules, including \textit{FloatingSE}, are built around the
OpenMDAO library, an open-source high-performance computing platform for
systems analysis and multidisciplinary optimization, written in Python
\citep{openmdao}.  Due to the structure of OpenMDAO, and modular design
of WISDEM, individual modules can be exercised individually or in unison
for turbine or plant level studies.  This feature also applies to
\textit{FloatingSE}, in that module-specific optimizations of a floating
substructure and its anchoring and mooring systems can be executed while
treating the rest of the turbine, plant, and operational strategy as
static.  Alternatively, \textit{FloatingSE} can be linked to other
WISDEM modules to execute turbine-level and/or plant-level tradeoff and
optimization studies.

This document serves as both a User Manual and Theory Guide for
\textit{FloatingSE}.  An overview of the package contents is in Section
\ref{sec:package} and substructure geometry parameterization in Section
\ref{sec:geom}.  With this foundation, the underlying theory of
\textit{FloatingSE's} methodology is explained in Section
\ref{sec:theory}.  This helps to understand the analysis execution flow
described in Section \ref{sec:exec} and the setup of design variables
and constraints for optimization problems in Section \ref{sec:opt}.
Finally, some discussion of how \textit{FloatingSE} can be linked to
other WISDEM modules via a high-level OpenMDAO Group is described in
Section \ref{sec:other}.
